\documentclass[]{article}
\usepackage{lmodern}
\usepackage{amssymb,amsmath}
\usepackage{ifxetex,ifluatex}
\usepackage{fixltx2e} % provides \textsubscript
\ifnum 0\ifxetex 1\fi\ifluatex 1\fi=0 % if pdftex
  \usepackage[T1]{fontenc}
  \usepackage[utf8]{inputenc}
\else % if luatex or xelatex
  \ifxetex
    \usepackage{mathspec}
  \else
    \usepackage{fontspec}
  \fi
  \defaultfontfeatures{Ligatures=TeX,Scale=MatchLowercase}
\fi
% use upquote if available, for straight quotes in verbatim environments
\IfFileExists{upquote.sty}{\usepackage{upquote}}{}
% use microtype if available
\IfFileExists{microtype.sty}{%
\usepackage{microtype}
\UseMicrotypeSet[protrusion]{basicmath} % disable protrusion for tt fonts
}{}
\usepackage[margin=1in]{geometry}
\usepackage{hyperref}
\hypersetup{unicode=true,
            pdftitle={Twitter Multilingual Sentiment Analysis},
            pdfauthor={César Villamizar, José de Melo},
            pdfborder={0 0 0},
            breaklinks=true}
\urlstyle{same}  % don't use monospace font for urls
\usepackage{color}
\usepackage{fancyvrb}
\newcommand{\VerbBar}{|}
\newcommand{\VERB}{\Verb[commandchars=\\\{\}]}
\DefineVerbatimEnvironment{Highlighting}{Verbatim}{commandchars=\\\{\}}
% Add ',fontsize=\small' for more characters per line
\usepackage{framed}
\definecolor{shadecolor}{RGB}{248,248,248}
\newenvironment{Shaded}{\begin{snugshade}}{\end{snugshade}}
\newcommand{\KeywordTok}[1]{\textcolor[rgb]{0.13,0.29,0.53}{\textbf{{#1}}}}
\newcommand{\DataTypeTok}[1]{\textcolor[rgb]{0.13,0.29,0.53}{{#1}}}
\newcommand{\DecValTok}[1]{\textcolor[rgb]{0.00,0.00,0.81}{{#1}}}
\newcommand{\BaseNTok}[1]{\textcolor[rgb]{0.00,0.00,0.81}{{#1}}}
\newcommand{\FloatTok}[1]{\textcolor[rgb]{0.00,0.00,0.81}{{#1}}}
\newcommand{\ConstantTok}[1]{\textcolor[rgb]{0.00,0.00,0.00}{{#1}}}
\newcommand{\CharTok}[1]{\textcolor[rgb]{0.31,0.60,0.02}{{#1}}}
\newcommand{\SpecialCharTok}[1]{\textcolor[rgb]{0.00,0.00,0.00}{{#1}}}
\newcommand{\StringTok}[1]{\textcolor[rgb]{0.31,0.60,0.02}{{#1}}}
\newcommand{\VerbatimStringTok}[1]{\textcolor[rgb]{0.31,0.60,0.02}{{#1}}}
\newcommand{\SpecialStringTok}[1]{\textcolor[rgb]{0.31,0.60,0.02}{{#1}}}
\newcommand{\ImportTok}[1]{{#1}}
\newcommand{\CommentTok}[1]{\textcolor[rgb]{0.56,0.35,0.01}{\textit{{#1}}}}
\newcommand{\DocumentationTok}[1]{\textcolor[rgb]{0.56,0.35,0.01}{\textbf{\textit{{#1}}}}}
\newcommand{\AnnotationTok}[1]{\textcolor[rgb]{0.56,0.35,0.01}{\textbf{\textit{{#1}}}}}
\newcommand{\CommentVarTok}[1]{\textcolor[rgb]{0.56,0.35,0.01}{\textbf{\textit{{#1}}}}}
\newcommand{\OtherTok}[1]{\textcolor[rgb]{0.56,0.35,0.01}{{#1}}}
\newcommand{\FunctionTok}[1]{\textcolor[rgb]{0.00,0.00,0.00}{{#1}}}
\newcommand{\VariableTok}[1]{\textcolor[rgb]{0.00,0.00,0.00}{{#1}}}
\newcommand{\ControlFlowTok}[1]{\textcolor[rgb]{0.13,0.29,0.53}{\textbf{{#1}}}}
\newcommand{\OperatorTok}[1]{\textcolor[rgb]{0.81,0.36,0.00}{\textbf{{#1}}}}
\newcommand{\BuiltInTok}[1]{{#1}}
\newcommand{\ExtensionTok}[1]{{#1}}
\newcommand{\PreprocessorTok}[1]{\textcolor[rgb]{0.56,0.35,0.01}{\textit{{#1}}}}
\newcommand{\AttributeTok}[1]{\textcolor[rgb]{0.77,0.63,0.00}{{#1}}}
\newcommand{\RegionMarkerTok}[1]{{#1}}
\newcommand{\InformationTok}[1]{\textcolor[rgb]{0.56,0.35,0.01}{\textbf{\textit{{#1}}}}}
\newcommand{\WarningTok}[1]{\textcolor[rgb]{0.56,0.35,0.01}{\textbf{\textit{{#1}}}}}
\newcommand{\AlertTok}[1]{\textcolor[rgb]{0.94,0.16,0.16}{{#1}}}
\newcommand{\ErrorTok}[1]{\textcolor[rgb]{0.64,0.00,0.00}{\textbf{{#1}}}}
\newcommand{\NormalTok}[1]{{#1}}
\usepackage{graphicx,grffile}
\makeatletter
\def\maxwidth{\ifdim\Gin@nat@width>\linewidth\linewidth\else\Gin@nat@width\fi}
\def\maxheight{\ifdim\Gin@nat@height>\textheight\textheight\else\Gin@nat@height\fi}
\makeatother
% Scale images if necessary, so that they will not overflow the page
% margins by default, and it is still possible to overwrite the defaults
% using explicit options in \includegraphics[width, height, ...]{}
\setkeys{Gin}{width=\maxwidth,height=\maxheight,keepaspectratio}
\IfFileExists{parskip.sty}{%
\usepackage{parskip}
}{% else
\setlength{\parindent}{0pt}
\setlength{\parskip}{6pt plus 2pt minus 1pt}
}
\setlength{\emergencystretch}{3em}  % prevent overfull lines
\providecommand{\tightlist}{%
  \setlength{\itemsep}{0pt}\setlength{\parskip}{0pt}}
\setcounter{secnumdepth}{0}
% Redefines (sub)paragraphs to behave more like sections
\ifx\paragraph\undefined\else
\let\oldparagraph\paragraph
\renewcommand{\paragraph}[1]{\oldparagraph{#1}\mbox{}}
\fi
\ifx\subparagraph\undefined\else
\let\oldsubparagraph\subparagraph
\renewcommand{\subparagraph}[1]{\oldsubparagraph{#1}\mbox{}}
\fi

%%% Use protect on footnotes to avoid problems with footnotes in titles
\let\rmarkdownfootnote\footnote%
\def\footnote{\protect\rmarkdownfootnote}

%%% Change title format to be more compact
\usepackage{titling}

% Create subtitle command for use in maketitle
\providecommand{\subtitle}[1]{
  \posttitle{
    \begin{center}\large#1\end{center}
    }
}

\setlength{\droptitle}{-2em}

  \title{Twitter Multilingual Sentiment Analysis}
    \pretitle{\vspace{\droptitle}\centering\huge}
  \posttitle{\par}
    \author{César Villamizar, José de Melo}
    \preauthor{\centering\large\emph}
  \postauthor{\par}
      \predate{\centering\large\emph}
  \postdate{\par}
    \date{April 26, 2019}


\begin{document}
\maketitle

\subsection{Context and data
questions}\label{context-and-data-questions}

The purpose of this document is to present a machine learning approach
to classify tweets in English and Portuguese using machine learning
techniques for sentiment analysis.

We will be focusing on a topic that has been on the international news
since the beginning of the year
\href{https://en.wikipedia.org/wiki/2019_Venezuelan_presidential_crisis}{``the
Venezuelan presidential crisis''}.

Venezuela is a democratic country and Venezuelans have officially
re-elected Nicolas Maduro as their president in 2018. However, the
results of this election has been deemed as fraudulent by opposition
parties and multiple countries. With the support of the Venezuelan
National Assembly and the international community, Juan Guaidó was
declared acting president of the nation since 10 January 2019.

The main questions we want to answer is : Are the tweets related to this
topic and posted in April 2019 by English and Portuguese speakers mostly
positive or negative? This is in machine learning classification problem
using text mining to classify the opinion of each tweet. Are these
opinions different depending on the location and/or language?

\subsection{Methodology}\label{methodology}

There are many examples of how to do sentiment analysis and for the
purpose of this exercise we intend to take advantage of the work
performed by other authors, mainly in the pre-classification of tweets
by sentiment.

In our research we encounter two main datasets with tweets
pre-classified as positive or negative in English and Portuguese. The
English dataset contains 1.6 million tweets and the Portuguese dataset
has 500K tweets.

Since most of the work on twitter sentiment analysis has been done in
English, we aim to replicate the steps of the feature extraction and
classification process, and apply a similar approach to the Portuguese
tweets.

The main steps for this project are:

\begin{enumerate}
\def\labelenumi{\arabic{enumi}.}
\tightlist
\item
  Extracting data related to the Venezuelan crisis
\end{enumerate}

\begin{itemize}
\tightlist
\item
  1.1 Classification of Tweets in English
\item
  1.2 Classification of Tweets in Portuguese
\end{itemize}

\begin{enumerate}
\def\labelenumi{\arabic{enumi}.}
\setcounter{enumi}{1}
\tightlist
\item
  Explore twitter labelled data
\end{enumerate}

\begin{itemize}
\tightlist
\item
  2.1 English labelled Tweets
\item
  3.2 Portuguese labelled Tweets
\end{itemize}

\begin{enumerate}
\def\labelenumi{\arabic{enumi}.}
\setcounter{enumi}{2}
\tightlist
\item
  Building Classificaion Models for Labelled Data
\end{enumerate}

\begin{itemize}
\tightlist
\item
  2.1 English (Pre-processing, Feature extraction and Classification
  Model)
\item
  2.2 Portuguese (Pre-processing, Feature extraction and Classification
  Model)
\end{itemize}

\begin{enumerate}
\def\labelenumi{\arabic{enumi}.}
\setcounter{enumi}{3}
\tightlist
\item
  Predictions and analysis of results
\item
  Lessons learned and next steps
\end{enumerate}

\subsection{1. Extracting data related to the Venezuelan
crisis}\label{extracting-data-related-to-the-venezuelan-crisis}

The sentiment classification will be implemented through four steps:
data extraction, pre-processing, feature selection, and classification
using the classification models trained with labelled data.

We start by loading the necessary packages for the project:

\begin{Shaded}
\begin{Highlighting}[]
\CommentTok{#LOAD PACKAGES----------------}

\KeywordTok{library}\NormalTok{(caret)}
\KeywordTok{library}\NormalTok{(dplyr)}
\KeywordTok{library}\NormalTok{(ggplot2)}
\KeywordTok{library}\NormalTok{(magrittr)}
\KeywordTok{library}\NormalTok{(tm)}
\KeywordTok{library}\NormalTok{(twitteR)}
\KeywordTok{library}\NormalTok{(purrr)}
\KeywordTok{library}\NormalTok{(plyr)}
\end{Highlighting}
\end{Shaded}

We proceed to extract tweets in both languages using the authentication
credentials obtained from the Twitter API, . The \texttt{twitteR}
package allows to use parameters to narrow the search for specific
location and language.

\begin{Shaded}
\begin{Highlighting}[]
\CommentTok{# AUTHENTICATION FOR THE TWITTER API----------------------------------------}

\NormalTok{consumerKey =}\StringTok{ "FU2sd3IX4CJDHfq62HjnSn7PF"}  
\NormalTok{consumerSecret =}\StringTok{ "iQkD1QsyRK0uqhAGzdzOoBOyzUxEfzAE79InkvhF0tzRu6IX85"}
\NormalTok{accessToken =}\StringTok{ "110476495-PM2CXoqDMAzUJriW7ZdCC1Mc7DbuSNZAWpyh6tJU"}
\NormalTok{accessSecret =}\StringTok{ "Nu4op1YW4IWDQLtVTXSZtAlUmmInNmn1NtjKGLJLByDpG"}
\KeywordTok{options}\NormalTok{(}\DataTypeTok{httr_oauth_cache=}\OtherTok{TRUE}\NormalTok{)}

\KeywordTok{setup_twitter_oauth}\NormalTok{(}\DataTypeTok{consumer_key =} \NormalTok{consumerKey, }\DataTypeTok{consumer_secret =} \NormalTok{consumerSecret,}
                    \DataTypeTok{access_token =} \NormalTok{accessToken, }\DataTypeTok{access_secret =} \NormalTok{accessSecret)}
\end{Highlighting}
\end{Shaded}

First we start with the tweets posted in English.

\begin{Shaded}
\begin{Highlighting}[]
\CommentTok{# CAPTURING TWEETS --------------------------------------------------}

\NormalTok{nbrTwts <-}\StringTok{ }\DecValTok{9000} \CommentTok{#limit is 10K every 15 minutes}

\NormalTok{vzltwts <-}\StringTok{ }\KeywordTok{c}\NormalTok{(}\StringTok{"#Venezuela"}\NormalTok{, }\StringTok{"#MaduroRegime"}\NormalTok{, }
             \StringTok{"@NicolasMaduro"}\NormalTok{, }\StringTok{"@maduro_en"}\NormalTok{,}
             \StringTok{"#VenezuelaLibre"}\NormalTok{, }\StringTok{"@jguaido"}\NormalTok{)}


\NormalTok{vzltwts <-}\StringTok{ }\KeywordTok{paste0}\NormalTok{(vzltwts,}\DataTypeTok{collapse =} \StringTok{" OR "}\NormalTok{)}

\CommentTok{# make a string for the days}
\NormalTok{days <-}\StringTok{ }\KeywordTok{as.character}\NormalTok{(}\KeywordTok{seq}\NormalTok{(}\KeywordTok{as.Date}\NormalTok{(}\StringTok{"2019-04-12"}\NormalTok{), }\KeywordTok{Sys.Date}\NormalTok{(),}\DataTypeTok{by =} \DecValTok{1}\NormalTok{)) }\CommentTok{# We can only go back 10 days with the type of account we have}


\CommentTok{#Extract Tweets in English posted in North America}
\NormalTok{tweetList_EN <-}\StringTok{ }\KeywordTok{lapply}\NormalTok{(}\KeywordTok{seq_along}\NormalTok{(days), function(x)\{}
  
  \NormalTok{res_en <-}\StringTok{ }\KeywordTok{searchTwitter}\NormalTok{(vzltwts,}
                \DataTypeTok{n =} \NormalTok{nbrTwts ,}
                \DataTypeTok{lang =}\StringTok{"en"}\NormalTok{,}
                \DataTypeTok{since =} \NormalTok{days[[x]],}
                \DataTypeTok{until =} \NormalTok{days[[x]],}
                \DataTypeTok{geocode =} \StringTok{'37.09024,-95.712891,3000km'}\NormalTok{) }\CommentTok{#North America}
  
  \KeywordTok{print}\NormalTok{(}\StringTok{"Tweets fetched successfully. Going to sleep now..."}\NormalTok{)}
  \KeywordTok{Sys.sleep}\NormalTok{(}\DataTypeTok{time =} \DecValTok{1100}\NormalTok{)}
  \KeywordTok{print}\NormalTok{(}\StringTok{"Return"}\NormalTok{)}
  \KeywordTok{return}\NormalTok{(res_en)}
\NormalTok{\})}

\KeywordTok{saveRDS} \NormalTok{(tweetList_EN, }\StringTok{"../helpers/data_extraction/tweetList_EN.RDS"} \NormalTok{)}
\end{Highlighting}
\end{Shaded}

And we continue with the twweets in portugese.

\begin{Shaded}
\begin{Highlighting}[]
\CommentTok{# CAPTURING TWEETS --------------------------------------------------}

\CommentTok{#Extract Tweets in Portuguese posted in South America}
\NormalTok{tweetList_PT <-}\StringTok{ }\KeywordTok{searchTwitter}\NormalTok{(vzltwts,}
                \DataTypeTok{n =} \NormalTok{nbrTwts,}
                \DataTypeTok{lang =}\StringTok{"pt"}\NormalTok{,}
                \DataTypeTok{since =} \StringTok{"2014-04-15"}\NormalTok{,}
                \DataTypeTok{geocode =} \StringTok{'-9.138173,-55.383780,2000km'}\NormalTok{) }\CommentTok{#South America}
  

\KeywordTok{saveRDS} \NormalTok{(tweetList_PT, }\StringTok{"../helpers/data_extraction/tweetList_PT.RDS"} \NormalTok{)}
\end{Highlighting}
\end{Shaded}

As we obtained a list in the previous process, we need to covert it to a
data frame.

\begin{Shaded}
\begin{Highlighting}[]
\CommentTok{#Convert the English tweets as a data.frame}
\NormalTok{ven_en <-}\KeywordTok{Reduce}\NormalTok{(}\StringTok{`}\DataTypeTok{c}\StringTok{`}\NormalTok{, tweetList_EN)}

\NormalTok{ven_en_df <-}\StringTok{ }\KeywordTok{tbl_df}\NormalTok{(}\KeywordTok{map_df}\NormalTok{(ven_en, as.data.frame))}

\KeywordTok{saveRDS}\NormalTok{(ven_en_df, }\StringTok{"../helpers/data_extraction/ven_en_df.RDS"} \NormalTok{)}


\CommentTok{#Convert the Portuguese tweets as a data.frame}
\NormalTok{ven_pt <-}\KeywordTok{Reduce}\NormalTok{(}\StringTok{`}\DataTypeTok{c}\StringTok{`}\NormalTok{, tweetList_PT)}

\NormalTok{ven_pt_df <-}\StringTok{ }\KeywordTok{tbl_df}\NormalTok{(}\KeywordTok{map_df}\NormalTok{(ven_pt, as.data.frame))}

\KeywordTok{saveRDS}\NormalTok{(ven_pt_df, }\StringTok{"../helpers/data_extraction/ven_pt_df.RDS"} \NormalTok{)}
\end{Highlighting}
\end{Shaded}

After exploring the data, we found that there is duplication in the data
extraction.

\begin{Shaded}
\begin{Highlighting}[]
\NormalTok{ven_en_df <-}\StringTok{ }\KeywordTok{readRDS}\NormalTok{(}\StringTok{"../helpers/data_extraction/ven_en_df.RDS"} \NormalTok{)}
\NormalTok{ven_pt_df <-}\StringTok{ }\KeywordTok{readRDS}\NormalTok{(}\StringTok{"../helpers/data_extraction/ven_pt_df.RDS"} \NormalTok{)}

\NormalTok{############### English dataset ##############}
\CommentTok{#Select the variables we need to combine with the labelled data}
\NormalTok{ven_en_df <-}\StringTok{ }\NormalTok{ven_en_df %>%}\StringTok{ }
\StringTok{  }\KeywordTok{select}\NormalTok{(id, text)}

\CommentTok{#Create and index for duplicated rows}
\NormalTok{dup <-}\StringTok{ }\KeywordTok{duplicated}\NormalTok{(ven_en_df$text)}

\CommentTok{#Remove duplicates from the dataset}
\NormalTok{ven_en_df <-}\StringTok{ }\NormalTok{ven_en_df[!dup, ]}

\NormalTok{############### Portuguese dataset ##############}

\NormalTok{ven_pt_df <-}\StringTok{ }\NormalTok{ven_pt_df %>%}\StringTok{ }
\StringTok{  }\KeywordTok{select}\NormalTok{(id, text)}

\CommentTok{#Create and index for duplicated rows}
\NormalTok{dup_pt <-}\StringTok{ }\KeywordTok{duplicated}\NormalTok{(ven_pt_df$text)}

\CommentTok{#Remove duplicates from the dataset}
\NormalTok{ven_pt_df <-}\StringTok{ }\NormalTok{ven_pt_df[!dup_pt, ]}

\CommentTok{#Change the column name tweet_text to text}

\NormalTok{ven_pt_df <-}\StringTok{ }\NormalTok{ven_pt_df  %>%}\StringTok{ }
\StringTok{  }\NormalTok{dplyr::}\KeywordTok{rename}\NormalTok{(}\DataTypeTok{tweet_text =} \NormalTok{text)}

\KeywordTok{str}\NormalTok{(ven_pt_df)}
\end{Highlighting}
\end{Shaded}

\subsection{2.Explore classified data.}\label{explore-classified-data.}

\subsubsection{2.1 English labelled
Tweets}\label{english-labelled-tweets}

The English dataset was created by Go, Bhayani \& Huang {[}1{]} and it
contains tweets extracted between August and September 2018. The tweets
were labelled as positive and negative using a distant classification
supervision method. The dataset includes the classification positive and
negative, the tweet id, date, flag, user and the text of the tweet.

\begin{Shaded}
\begin{Highlighting}[]
\CommentTok{#IMPORT ENGLISH LABELLED DATASET ----}
\NormalTok{TwitTerData_EN <-}\StringTok{ }\KeywordTok{read.csv}\NormalTok{(}\StringTok{"../data/labeled_en/twitterSentiment.csv"}\NormalTok{, }\DataTypeTok{stringsAsFactors =} \NormalTok{F, }\DataTypeTok{header =} \NormalTok{F)}

\CommentTok{#Set names of the variables}
\NormalTok{colnames <-}\StringTok{ }\KeywordTok{c}\NormalTok{(}\StringTok{"target"}\NormalTok{, }\StringTok{"id"}\NormalTok{, }\StringTok{"date"}\NormalTok{, }\StringTok{"flag"}\NormalTok{, }\StringTok{"user"}\NormalTok{, }\StringTok{"text"}\NormalTok{)}
\KeywordTok{names}\NormalTok{(TwitTerData_EN) <-colnames}
 
\CommentTok{#Save data as RDS file}
\KeywordTok{saveRDS}\NormalTok{(TwitTerData_EN,}\StringTok{"../data/labeled_en/twtr_dataset_en.RDS"}\NormalTok{) }
\end{Highlighting}
\end{Shaded}

The dataset has 50\% positive and 50\% negative labels, 0 for negative
tweets and 4 for positive tweets.

\begin{Shaded}
\begin{Highlighting}[]
\CommentTok{# Identify the frequency of classified sentiments }
\NormalTok{TwitTerData_EN <-}\StringTok{ }\KeywordTok{readRDS}\NormalTok{(}\StringTok{"../data/labeled_en/twtr_dataset_en.RDS"}\NormalTok{)}

\NormalTok{sentfrequ <-}\StringTok{ }\NormalTok{TwitTerData_EN %>%}
\StringTok{    }\NormalTok{dplyr::}\KeywordTok{select}\NormalTok{(target) %>%}\StringTok{ }
\StringTok{    }\NormalTok{dplyr::}\KeywordTok{count}\NormalTok{(target)}
  
\NormalTok{sentfrequ}
\end{Highlighting}
\end{Shaded}

We will convert this classification to 0 for negative and 1 for
positive, and creating a new column call \texttt{sentiment} to save the
classification.

\begin{Shaded}
\begin{Highlighting}[]
\CommentTok{#Convert sentiment to 0 negative and 1 positive}
\NormalTok{TwitTerData_EN <-}\StringTok{ }\NormalTok{TwitTerData_EN %>%}
\StringTok{  }\NormalTok{dplyr::}\KeywordTok{mutate}\NormalTok{(}\DataTypeTok{target =} \NormalTok{dplyr::}\KeywordTok{if_else}\NormalTok{(target ==}\StringTok{ }\DecValTok{0}\NormalTok{, }\DecValTok{0}\NormalTok{, }\DecValTok{1}\NormalTok{)) %>%}\StringTok{ }
\StringTok{  }\NormalTok{dplyr::}\KeywordTok{mutate}\NormalTok{(}\DataTypeTok{sentiment =} \KeywordTok{as.factor}\NormalTok{(target))}\CommentTok{# change data type from character to factor}

\KeywordTok{str}\NormalTok{(TwitTerData_EN)}
\end{Highlighting}
\end{Shaded}

\paragraph{1.2 Portuguese labelled
Tweets}\label{portuguese-labelled-tweets}

The Portuguese dataset was created by de Santiago {[}2{]} using a
similar approach than the one used by Go. et al. The dataset was
collected using 100 political terms and it also includes the tweet id,
the text, date and the labelled sentiment.

\begin{Shaded}
\begin{Highlighting}[]
\CommentTok{#IMPORT PORTUGUESE LABELLED DATASET ----}
\NormalTok{TwitTerData_PT <-}\StringTok{ }\KeywordTok{read.csv}\NormalTok{(}\StringTok{"../data/labeled_pt/Train500.csv"}\NormalTok{, }\DataTypeTok{stringsAsFactors =} \NormalTok{F)}



\CommentTok{#Convert sentiment to 0 negative and 1 positive}
\NormalTok{TwitTerData_PT <-}\StringTok{ }\NormalTok{TwitTerData_PT %>%}
\StringTok{    }\NormalTok{dplyr::}\KeywordTok{mutate}\NormalTok{(}\DataTypeTok{sentiment =} \KeywordTok{as.factor}\NormalTok{(sentiment))}\CommentTok{# change data type from character to factor}


\KeywordTok{str}\NormalTok{(TwitTerData_PT)}

\KeywordTok{saveRDS}\NormalTok{(TwitTerData_PT, }\StringTok{"../data/labeled_pt/TwitTerData_PT.RDS"}\NormalTok{)}
\end{Highlighting}
\end{Shaded}

\begin{Shaded}
\begin{Highlighting}[]
\CommentTok{# Identify the frequency of classified sentiments }
\NormalTok{TwitTerData_PT<-}\StringTok{ }\KeywordTok{readRDS}\NormalTok{(}\StringTok{"../data/labeled_pt/twtr_dataset_pt.RDS"}\NormalTok{)}


\NormalTok{sentfrequ <-}\StringTok{ }\NormalTok{TwitTerData_PT %>%}
\StringTok{    }\NormalTok{dplyr::}\KeywordTok{select}\NormalTok{(sentiment) %>%}\StringTok{ }
\StringTok{    }\NormalTok{dplyr::}\KeywordTok{count}\NormalTok{(sentiment)}
  
\NormalTok{sentfrequ}
\end{Highlighting}
\end{Shaded}

The dataset has as well a distribution of 50\% positive and 50\%
negative labels, 0 for negative tweets and 1 for positive tweets.

\subsubsection{2. Building Classificaion Models for Labelled
Data}\label{building-classificaion-models-for-labelled-data}

\paragraph{2.1 English (Pre-processing, Feature extraction and
Classification
Model)}\label{english-pre-processing-feature-extraction-and-classification-model}

The first step is to combine the extracted and labelled dataset to
create a common corpus

\begin{Shaded}
\begin{Highlighting}[]
\NormalTok{TwitTerData_EN <-}\StringTok{ }\NormalTok{plyr::}\KeywordTok{rbind.fill}\NormalTok{(TwitTerData_EN, ven_en_df)}
\end{Highlighting}
\end{Shaded}

We start by creating a function to remove unnecessary information from
the tweet and then apply it to the dataset, e.g.~removing the stop
words, white spaces, repeating words, emoticons and \#hash tags.

\begin{Shaded}
\begin{Highlighting}[]
\NormalTok{ProcessTweets <-}\StringTok{ }\NormalTok{function (x) \{}
  
  \CommentTok{# select the column that has the tweet text}
  \NormalTok{tweet <-}\StringTok{ }\NormalTok{x$text}
  \CommentTok{# remove @user tagged in the tweet}
  \NormalTok{tweet <-}\StringTok{ }\KeywordTok{gsub}\NormalTok{(}\StringTok{"@}\CharTok{\textbackslash{}\textbackslash{}}\StringTok{S+"}\NormalTok{, }\StringTok{" "}\NormalTok{, tweet)}
  \CommentTok{# remove #hashtags in the text}
  \NormalTok{tweet <-}\StringTok{ }\KeywordTok{gsub}\NormalTok{(}\StringTok{"#}\CharTok{\textbackslash{}\textbackslash{}}\StringTok{S+"}\NormalTok{, }\StringTok{" "}\NormalTok{, tweet)}
  \CommentTok{# remove links}
  \NormalTok{tweet <-}\StringTok{ }\KeywordTok{gsub}\NormalTok{(}\StringTok{"http}\CharTok{\textbackslash{}\textbackslash{}}\StringTok{S+"}\NormalTok{, }\StringTok{" "}\NormalTok{, tweet)}
  \CommentTok{# remove punctuation }
  \NormalTok{tweet <-}\StringTok{ }\KeywordTok{gsub}\NormalTok{(}\StringTok{"[[:punct:]]"}\NormalTok{, }\StringTok{" "}\NormalTok{, tweet)}
  \CommentTok{# remove numbers e.g. times}
  \NormalTok{tweet <-}\StringTok{ }\KeywordTok{gsub}\NormalTok{(}\StringTok{"[[:digit:]]"}\NormalTok{, }\StringTok{" "}\NormalTok{, tweet)}
  \CommentTok{# normalize words -> convert every word to lowercase}
  \NormalTok{tweet <-}\StringTok{ }\KeywordTok{tolower}\NormalTok{(tweet)}
  \CommentTok{# remove stop words using the tm stopword lexicon}
  \NormalTok{tweet <-}\StringTok{ }\NormalTok{tm::}\KeywordTok{removeWords}\NormalTok{(tweet, tm::}\KeywordTok{stopwords}\NormalTok{(}\StringTok{"English"}\NormalTok{))}
  \CommentTok{# remove stop words using the tm stopword lexicon}
  \NormalTok{tweet <-}\StringTok{ }\NormalTok{tm::}\KeywordTok{removeWords}\NormalTok{(tweet, tm::}\KeywordTok{stopwords}\NormalTok{(}\StringTok{"English"}\NormalTok{))}
  \CommentTok{#Stem document}
  \NormalTok{tweet <-}\StringTok{ }\NormalTok{tm::}\KeywordTok{stemDocument}\NormalTok{(tweet, }\DataTypeTok{language =} \StringTok{"en"}\NormalTok{)}
  \CommentTok{# replace letters that have more than 3 repetition}
  \CommentTok{# this deals with words like looooong and soooooooo and change them to long and so   }
  \NormalTok{regExp <-}\StringTok{ }\KeywordTok{sapply}\NormalTok{(letters, function(x) }\KeywordTok{paste0}\NormalTok{(}\KeywordTok{paste0}\NormalTok{(}\KeywordTok{rep}\NormalTok{(x, }\DecValTok{3}\NormalTok{), }\DataTypeTok{collapse=}\StringTok{""}\NormalTok{), }\StringTok{"+"}\NormalTok{))}
  \NormalTok{for(i in }\KeywordTok{seq_along}\NormalTok{(regExp))\{}
    \NormalTok{tweet <-}\StringTok{ }\KeywordTok{gsub}\NormalTok{(}\DataTypeTok{x =} \NormalTok{tweet, }\DataTypeTok{pattern =} \NormalTok{regExp[[i]], }\DataTypeTok{replacement =} \NormalTok{letters[[i]])}
  \NormalTok{\}}
  \CommentTok{#deals with special characters in the text }
  \NormalTok{tweet<-}\StringTok{ }\KeywordTok{iconv}\NormalTok{(tweet, }\StringTok{"latin1"}\NormalTok{, }\StringTok{"ASCII"}\NormalTok{, }\StringTok{""}\NormalTok{)}
  \CommentTok{# remove words less than 2 characters}
  \NormalTok{tweet <-}\StringTok{ }\KeywordTok{gsub}\NormalTok{(}\StringTok{"(}\CharTok{\textbackslash{}\textbackslash{}}\StringTok{b)?}\CharTok{\textbackslash{}\textbackslash{}}\StringTok{w\{1,2\}(}\CharTok{\textbackslash{}\textbackslash{}}\StringTok{b)?"}\NormalTok{, }\StringTok{" "}\NormalTok{, tweet)}
  \CommentTok{#Remove leading and/or trailing whitespace and other white spaces}
  \NormalTok{tweet <-}\KeywordTok{trimws}\NormalTok{(tweet)}
  \NormalTok{tweet <-}\StringTok{ }\KeywordTok{gsub}\NormalTok{(}\StringTok{"[ }\CharTok{\textbackslash{}t}\StringTok{]\{2,\}"}\NormalTok{, }\StringTok{" "}\NormalTok{, tweet)}
  \CommentTok{# bind processed tweets with the data frame}
  \NormalTok{x <-}\StringTok{ }\KeywordTok{data.frame} \NormalTok{(}\KeywordTok{cbind.data.frame}\NormalTok{(}\DataTypeTok{id =} \NormalTok{x$id,}
                         \DataTypeTok{text =} \NormalTok{x$text,}
                         \DataTypeTok{sentiment =} \NormalTok{x$sentiment,}
                         \DataTypeTok{text_sa =} \NormalTok{tweet))                             }
  \KeywordTok{return}\NormalTok{(x)}
\NormalTok{\}}
\end{Highlighting}
\end{Shaded}

Now we apply the function to the dataset to process the tweets. Since
this is a large dataset and depending on the configuration of the
computer it can take a considerable amount of time to process.
Throughout the code we are saving the results as RDS files in case there
is a need to skip some of the steps.

\begin{Shaded}
\begin{Highlighting}[]
\NormalTok{TwitTerData_EN_Processed <-}\StringTok{ }\KeywordTok{ProcessTweets}\NormalTok{(TwitTerData_EN)}

\KeywordTok{str}\NormalTok{(TwitTerData_EN_Processed)}
\end{Highlighting}
\end{Shaded}

The variable \texttt{text\_sa} was converted as a factor and it needs to
be a character vector

\begin{Shaded}
\begin{Highlighting}[]
\NormalTok{TwitTerData_EN_Processed$text_sa <-}\StringTok{ }\KeywordTok{as.character}\NormalTok{(TwitTerData_EN_Processed$text_sa)}

\CommentTok{#Filter out tweets that have no words associated after the initial processing}
\NormalTok{TwitTerData_EN_Processed <-}\StringTok{ }\NormalTok{TwitTerData_EN_Processed %>%}\StringTok{ }
\StringTok{  }\KeywordTok{filter}\NormalTok{(text_sa !=}\StringTok{ ""}\NormalTok{)}

\KeywordTok{saveRDS}\NormalTok{(TwitTerData_EN_Processed, }\StringTok{"../data/labeled_en/TwitTerData_EN_Processed.RDS"}\NormalTok{)}
  
\KeywordTok{str}\NormalTok{(TwitTerData_EN_Processed)}
\end{Highlighting}
\end{Shaded}

The next step is to extract the features of the tweets by converting the
processed text to a document text matrix using the \texttt{tm} package.
We will also remove tweets that have no text after the pre-processing.

\begin{Shaded}
\begin{Highlighting}[]
\CommentTok{# Generate document text matrix}

\NormalTok{TwitTerData_EN_Processed <-}\StringTok{ }\KeywordTok{readRDS}\NormalTok{(}\StringTok{"../data/labeled_en/TwitTerData_EN_Processed.RDS"}\NormalTok{)}

\NormalTok{ENTweetCorpus <-}\StringTok{ }\NormalTok{tm::}\KeywordTok{Corpus}\NormalTok{(tm::}\KeywordTok{VectorSource}\NormalTok{(TwitTerData_EN_Processed$text_sa))}

\NormalTok{dtm_EN <-}\StringTok{ }\NormalTok{tm::}\KeywordTok{DocumentTermMatrix}\NormalTok{(ENTweetCorpus)}

\KeywordTok{saveRDS}\NormalTok{(dtm_EN, }\StringTok{"../data/labeled_en/dtm_EN.RDS"}\NormalTok{)}

\CommentTok{#remove sparse terms}
\NormalTok{dtm_EN_sparse <-}\StringTok{ }\NormalTok{tm::}\KeywordTok{removeSparseTerms}\NormalTok{(dtm_EN, .}\DecValTok{99}\NormalTok{)}

\KeywordTok{saveRDS}\NormalTok{(dtm_EN_sparse, }\StringTok{"../data/labeled_en/dtm_EN_sparse.RDS"}\NormalTok{)}
\end{Highlighting}
\end{Shaded}

\begin{Shaded}
\begin{Highlighting}[]
\CommentTok{#Need to increase memory size to complete this part because the configuration of my machine}
\KeywordTok{memory.limit}\NormalTok{(}\DataTypeTok{size=}\DecValTok{53000}\NormalTok{)}

\CommentTok{#convert dtm as a matrix}
\NormalTok{ENdtm <-}\StringTok{ }\KeywordTok{as.matrix}\NormalTok{(dtm_EN_sparse)}

\KeywordTok{saveRDS}\NormalTok{(ENdtm, }\StringTok{"../data/labeled_en/ENdtm.RDS"}\NormalTok{)}

\CommentTok{#convert matrix as data frame}
\NormalTok{ENdtmDF <-}\StringTok{ }\KeywordTok{as.data.frame}\NormalTok{(ENdtm)}

\CommentTok{#add sentiment target to the data frame}
\NormalTok{ENdtmDF_sa <-}\StringTok{ }\KeywordTok{cbind}\NormalTok{(ENdtmDF, }\DataTypeTok{sentiment_target =} \NormalTok{TwitTerData_EN_Processed$sentiment)}

\CommentTok{#confirm there are not Na values in the dataset}
\KeywordTok{anyNA}\NormalTok{(ENdtmDF_sa)}

\CommentTok{#save EN_dtm as an RDS}
\KeywordTok{saveRDS} \NormalTok{(ENdtmDF_sa, }\StringTok{"../data/labeled_en/ENdtmDF_sa.RDS"}\NormalTok{)}
\end{Highlighting}
\end{Shaded}

As we have now converted the dataset to a document text matrix for the
English tweets, the next step is to start building the classification
model. The classification model used in this project is based on the
Naive Bayes logistic algorithm. Because of machine configuration, we
will only use 30\% of the data and split it in 80/20 for training and
testing. We use the \texttt{caret} package to complete this section of
the process.

\begin{Shaded}
\begin{Highlighting}[]
\CommentTok{#ENdtmDF_sa <- readRDS("../data/labeled_en/ENdtmDF_sa.RDS") ## - > start here if you only want to run the model}

\CommentTok{#Filter out the dataset with the tweets extracted linked to the venezuelan conflict }
\NormalTok{holdout_en <-}\StringTok{ }\KeywordTok{is.na}\NormalTok{(ENdtmDF_sa$sentiment_target)}

\NormalTok{ENdtmDF_sa <-}\StringTok{ }\NormalTok{ENdtmDF_sa[!holdout_en, ]}

\CommentTok{# Extract sample of the dataset   }
\KeywordTok{set.seed}\NormalTok{(}\DecValTok{9876}\NormalTok{)}
\NormalTok{EN_index <-}\StringTok{ }\NormalTok{caret::}\KeywordTok{createDataPartition}\NormalTok{(ENdtmDF_sa$sentiment_target, }\DataTypeTok{p =} \FloatTok{0.25}\NormalTok{, }
                                       \DataTypeTok{list =} \OtherTok{FALSE}\NormalTok{, }
                                       \DataTypeTok{times =} \DecValTok{1}\NormalTok{)}

\NormalTok{ENdtmDF_sample <-}\StringTok{ }\NormalTok{ENdtmDF_sa[EN_index, ]}

\CommentTok{#confirm that the number of classification variables is distributed equally}
\KeywordTok{table}\NormalTok{(ENdtmDF_sample$sentiment)}
\end{Highlighting}
\end{Shaded}

The result is a sample consisting of 198,741 negative tweets and 198,597
positive tweets. This dataset will be split in training and testing.

\begin{Shaded}
\begin{Highlighting}[]
\CommentTok{#Split dataset in training (80%) and testing (20%)}
\KeywordTok{set.seed}\NormalTok{(}\DecValTok{2312}\NormalTok{)}
\NormalTok{EN_index_sample <-}\StringTok{ }\NormalTok{caret::}\KeywordTok{createDataPartition}\NormalTok{(ENdtmDF_sample$sentiment, }\DataTypeTok{p =} \FloatTok{0.8}\NormalTok{, }
                                             \DataTypeTok{list =} \OtherTok{FALSE}\NormalTok{, }
                                             \DataTypeTok{times =} \DecValTok{1}\NormalTok{)      }

\NormalTok{EN_training <-}\StringTok{ }\NormalTok{ENdtmDF_sample[EN_index_sample, ]}
\NormalTok{EN_test  <-}\StringTok{ }\NormalTok{ENdtmDF_sample[-EN_index_sample,]}

\CommentTok{#verify the distribution of the dataset }
\KeywordTok{table}\NormalTok{(EN_training$sentiment_target)}

    \CommentTok{#save EN_dtm as an RDS}
    \KeywordTok{saveRDS} \NormalTok{(EN_training, }\StringTok{"../data/labeled_en/EN_training.RDS"}\NormalTok{)}
    \KeywordTok{saveRDS} \NormalTok{(EN_test, }\StringTok{"../data/labeled_en/EN_test.RDS"}\NormalTok{)}
\end{Highlighting}
\end{Shaded}

The final step is to run the classification model, using the Bayesian
Generalized Linear Models for supervised machine learning. Please note
that this process can take up to an hour depending on the configuration
of your machine.

\begin{Shaded}
\begin{Highlighting}[]
\CommentTok{#Need to increase memory to run the process in my machine}
\KeywordTok{memory.limit}\NormalTok{(}\DataTypeTok{size=} \DecValTok{53000}\NormalTok{)    }

\CommentTok{# Training the model}
\NormalTok{StartTIme <-}\StringTok{ }\KeywordTok{Sys.time}\NormalTok{()}
\NormalTok{fit <-}\StringTok{ }\NormalTok{caret::}\KeywordTok{train}\NormalTok{(sentiment_target ~}\StringTok{ }\NormalTok{., }\DataTypeTok{data =} \NormalTok{EN_training, }\DataTypeTok{method =} \StringTok{'bayesglm'}\NormalTok{)}
\NormalTok{EndTime <-}\StringTok{ }\KeywordTok{Sys.time}\NormalTok{()}
\NormalTok{EndTime -}\StringTok{ }\NormalTok{StartTIme}

\CommentTok{#save the model}
\KeywordTok{saveRDS}\NormalTok{(fit, }\StringTok{"../data/labeled_en/fit.RDS"}\NormalTok{)}
\end{Highlighting}
\end{Shaded}

And test the model.

\begin{Shaded}
\begin{Highlighting}[]
\CommentTok{#Need to increase memory to run the process in my machine}
\KeywordTok{memory.limit}\NormalTok{(}\DataTypeTok{size=} \DecValTok{53000}\NormalTok{)    }

\NormalTok{EN_test <-}\StringTok{ }\KeywordTok{readRDS} \NormalTok{(}\StringTok{"../data/labeled_en/EN_test.RDS"}\NormalTok{)}
\NormalTok{fit <-}\StringTok{ }\KeywordTok{readRDS}\NormalTok{(}\StringTok{"../data/labeled_en/fit.RDS"}\NormalTok{)}


\CommentTok{# Testing the model}
\NormalTok{predict <-}\StringTok{ }\KeywordTok{predict}\NormalTok{(fit, }\DataTypeTok{newdata =} \NormalTok{EN_test)}


\KeywordTok{saveRDS} \NormalTok{(predict, }\StringTok{"../data/labeled_en/predict.RDS"}\NormalTok{)}
\end{Highlighting}
\end{Shaded}

\begin{Shaded}
\begin{Highlighting}[]
\CommentTok{# Verify the results}
\NormalTok{fit }

\CommentTok{#confirm the results}
\NormalTok{expected_EN <-}\StringTok{ }\NormalTok{EN_test$sentiment_target}

\NormalTok{results_EN <-}\StringTok{ }\KeywordTok{confusionMatrix}\NormalTok{(}\DataTypeTok{data =} \NormalTok{predict, }\DataTypeTok{reference =} \NormalTok{expected_EN)}

\NormalTok{results_EN}
\end{Highlighting}
\end{Shaded}

The accuracy of the results of the training and testing model is low,
only 66\%. We estimate that using more data to train the model can
increase the accuracy of the model. For the purpose of this project we
will apply this model to the data extraction of English tweets and
predict the outcome of the obtained level of accuracy.

\subsubsection{2.2 Portuguese (Pre-processing, Feature extraction and
Classification
Model)}\label{portuguese-pre-processing-feature-extraction-and-classification-model}

We'll combine the Portuguese extracted dataset with the labelled dataset
to create a common corpus

\begin{Shaded}
\begin{Highlighting}[]
\NormalTok{TwitTerData_PT <-}\StringTok{ }\NormalTok{plyr::}\KeywordTok{rbind.fill}\NormalTok{(TwitTerData_PT, ven_pt_df)}
\end{Highlighting}
\end{Shaded}

As we did with the English tweets, we will use a similar function to
remove words from the tweets in Portuguese. In addition to the process
of English tweets, we will also identify other stop words and special
characters.

\begin{Shaded}
\begin{Highlighting}[]
\NormalTok{ProcessTweets_PT <-}\StringTok{ }\NormalTok{function (x) \{}
  
  \CommentTok{# select the column that has the tweet text}
  \NormalTok{tweet <-}\StringTok{ }\NormalTok{x$tweet_text}
  \CommentTok{# remove @user tagged in the tweet}
  \NormalTok{tweet <-}\StringTok{ }\KeywordTok{gsub}\NormalTok{(}\StringTok{"@}\CharTok{\textbackslash{}\textbackslash{}}\StringTok{S+"}\NormalTok{, }\StringTok{" "}\NormalTok{, tweet)}
  \CommentTok{# remove #hashtags in the text}
  \NormalTok{tweet <-}\StringTok{ }\KeywordTok{gsub}\NormalTok{(}\StringTok{"#}\CharTok{\textbackslash{}\textbackslash{}}\StringTok{S+"}\NormalTok{, }\StringTok{" "}\NormalTok{, tweet)}
  \CommentTok{# remove links}
  \NormalTok{tweet <-}\StringTok{ }\KeywordTok{gsub}\NormalTok{(}\StringTok{"http}\CharTok{\textbackslash{}\textbackslash{}}\StringTok{S+"}\NormalTok{, }\StringTok{" "}\NormalTok{, tweet)}
  \CommentTok{# remove punctuation }
  \NormalTok{tweet <-}\StringTok{ }\KeywordTok{gsub}\NormalTok{(}\StringTok{"[[:punct:]]"}\NormalTok{, }\StringTok{" "}\NormalTok{, tweet)}
  \CommentTok{# remove numbers e.g. times}
  \NormalTok{tweet <-}\StringTok{ }\KeywordTok{gsub}\NormalTok{(}\StringTok{"[[:digit:]]"}\NormalTok{, }\StringTok{" "}\NormalTok{, tweet)}
  \CommentTok{# normalize words -> convert every word to lowercase}
  \NormalTok{tweet <-}\StringTok{ }\KeywordTok{tolower}\NormalTok{(tweet)}
  \CommentTok{# remove stop words using the tm stopword lexicon}
  \NormalTok{tweet <-}\StringTok{ }\NormalTok{tm::}\KeywordTok{removeWords}\NormalTok{(tweet, tm::}\KeywordTok{stopwords}\NormalTok{(}\StringTok{"portuguese"}\NormalTok{))}
  \CommentTok{# remove stop words using the tm stopword lexicon}
  \NormalTok{tweet <-}\StringTok{ }\NormalTok{tm::}\KeywordTok{removeWords}\NormalTok{(tweet, }\KeywordTok{c}\NormalTok{(}\StringTok{"pra"}\NormalTok{, }\StringTok{"vou"} \NormalTok{))}
  \CommentTok{# remove other simbols in the text}
  \NormalTok{tweet <-}\StringTok{ }\KeywordTok{gsub}\NormalTok{(}\StringTok{"-}\CharTok{\textbackslash{}\textbackslash{}}\StringTok{S+"}\NormalTok{, }\StringTok{" "}\NormalTok{, tweet)}
  \NormalTok{tweet <-}\StringTok{ }\KeywordTok{gsub}\NormalTok{(}\StringTok{"_}\CharTok{\textbackslash{}\textbackslash{}}\StringTok{S+"}\NormalTok{, }\StringTok{" "}\NormalTok{, tweet)}
  \CommentTok{#Stem document}
  \CommentTok{# this deals with words like brasiiiiill or orgulhooooooooooo and change them to brasil and orgulho}
   \NormalTok{regExp <-}\StringTok{ }\KeywordTok{sapply}\NormalTok{(letters, function(x) }\KeywordTok{paste0}\NormalTok{(}\KeywordTok{paste0}\NormalTok{(}\KeywordTok{rep}\NormalTok{(x, }\DecValTok{3}\NormalTok{), }\DataTypeTok{collapse=}\StringTok{""}\NormalTok{), }\StringTok{"+"}\NormalTok{))}
   \NormalTok{for(i in }\KeywordTok{seq_along}\NormalTok{(regExp))\{}
       \NormalTok{tweet <-}\StringTok{ }\KeywordTok{gsub}\NormalTok{(}\DataTypeTok{x =} \NormalTok{tweet, }\DataTypeTok{pattern =} \NormalTok{regExp[[i]], }\DataTypeTok{replacement =} \NormalTok{letters[[i]])}
    \NormalTok{\}}
  \CommentTok{#deals with special characters in the text }
  \NormalTok{tweet<-}\StringTok{ }\KeywordTok{iconv}\NormalTok{(tweet, }\StringTok{"latin1"}\NormalTok{, }\StringTok{"ASCII"}\NormalTok{, }\StringTok{""}\NormalTok{)}
  \CommentTok{#Sten words}
  \NormalTok{tweet <-}\StringTok{ }\NormalTok{tm::}\KeywordTok{stemDocument}\NormalTok{(tweet, }\DataTypeTok{language =} \StringTok{"pt"}\NormalTok{)}
  \CommentTok{# remove words less than 2 characters}
  \NormalTok{tweet <-}\StringTok{ }\KeywordTok{gsub}\NormalTok{(}\StringTok{"(}\CharTok{\textbackslash{}\textbackslash{}}\StringTok{b)?}\CharTok{\textbackslash{}\textbackslash{}}\StringTok{w\{1,2\}(}\CharTok{\textbackslash{}\textbackslash{}}\StringTok{b)?"}\NormalTok{, }\StringTok{" "}\NormalTok{, tweet)}
  \CommentTok{#Remove leading and/or trailing whitespace and other white spaces}
  \NormalTok{tweet <-}\KeywordTok{trimws}\NormalTok{(tweet)}
  \NormalTok{tweet <-}\StringTok{ }\KeywordTok{gsub}\NormalTok{(}\StringTok{"[ }\CharTok{\textbackslash{}t}\StringTok{]\{2,\}"}\NormalTok{, }\StringTok{" "}\NormalTok{, tweet)}
  \CommentTok{# bind processed tweets with the data frame}
 \NormalTok{x <-}\StringTok{ }\KeywordTok{data.frame} \NormalTok{(}\KeywordTok{cbind.data.frame}\NormalTok{(}\DataTypeTok{id =} \NormalTok{x$id,}
                         \DataTypeTok{tweet_text_ =} \NormalTok{x$tweet_text,}
                         \DataTypeTok{sentiment =} \NormalTok{x$sentiment,}
                          \DataTypeTok{text_sa =} \NormalTok{tweet))    }
                           
  \KeywordTok{return}\NormalTok{(x)}
\NormalTok{\}}
\end{Highlighting}
\end{Shaded}

Now we apply the function to the dataset to process the tweets in
Portuguese.

\begin{Shaded}
\begin{Highlighting}[]
\NormalTok{TwitTerData_PT_Processed <-}\StringTok{ }\KeywordTok{ProcessTweets_PT}\NormalTok{(TwitTerData_PT)}

\KeywordTok{str}\NormalTok{(TwitTerData_PT_Processed)}
\end{Highlighting}
\end{Shaded}

Remove tweets that have no text after the pre-processing step. The
variable \texttt{text\_sa} was converted as a factor and it needs to be
a character vector.

\begin{Shaded}
\begin{Highlighting}[]
\NormalTok{TwitTerData_PT_Processed$text_sa <-}\StringTok{ }\KeywordTok{as.character}\NormalTok{(TwitTerData_PT_Processed$text_sa)}

\CommentTok{#Filter out tweets that have no words associated after the initial processing}
\NormalTok{TwitTerData_PT_Processed <-}\StringTok{ }\NormalTok{TwitTerData_PT_Processed %>%}\StringTok{ }
\StringTok{  }\KeywordTok{filter}\NormalTok{(text_sa !=}\StringTok{ ""}\NormalTok{)}

\KeywordTok{saveRDS}\NormalTok{(TwitTerData_PT_Processed, }\StringTok{"../data/labeled_pt/TwitTerData_PT_Processed.RDS"}\NormalTok{)}
  
\KeywordTok{str}\NormalTok{(TwitTerData_PT_Processed)}
\end{Highlighting}
\end{Shaded}

The next step is to extract the features of the tweets by converting the
processed text to a document text matrix.

\begin{Shaded}
\begin{Highlighting}[]
\CommentTok{# Generate document text matrix}

\NormalTok{PT_TweetCorpus <-}\StringTok{ }\NormalTok{tm::}\KeywordTok{Corpus}\NormalTok{(tm::}\KeywordTok{VectorSource}\NormalTok{(TwitTerData_PT_Processed$text_sa))}

\NormalTok{dtm_PT <-}\StringTok{ }\NormalTok{tm::}\KeywordTok{DocumentTermMatrix}\NormalTok{(PT_TweetCorpus)}

\KeywordTok{saveRDS}\NormalTok{(dtm_PT, }\StringTok{"../data/labeled_pt/dtm_PT.RDS"}\NormalTok{)}

\CommentTok{#remove sparse terms}
\NormalTok{dtm_PT_sparse <-}\StringTok{ }\NormalTok{tm::}\KeywordTok{removeSparseTerms}\NormalTok{(dtm_PT, .}\DecValTok{99}\NormalTok{)}

\KeywordTok{saveRDS}\NormalTok{(dtm_PT_sparse, }\StringTok{"../data/labeled_pt/dtm_PT_sparse.RDS"}\NormalTok{)}
\end{Highlighting}
\end{Shaded}

\begin{Shaded}
\begin{Highlighting}[]
\CommentTok{#Need to increase memory size to complete this part because the configuration of my machine}
\KeywordTok{memory.limit}\NormalTok{(}\DataTypeTok{size=}\DecValTok{53000}\NormalTok{)}

\CommentTok{#convert dtm as a matrix}
\NormalTok{PTdtm <-}\StringTok{ }\KeywordTok{as.matrix}\NormalTok{(dtm_PT_sparse)}

\KeywordTok{saveRDS}\NormalTok{(PTdtm, }\StringTok{"../data/labeled_pt/PTdtm.RDS"}\NormalTok{)}

\CommentTok{#convert matrix as data frame}
\NormalTok{PTdtmDF <-}\StringTok{ }\KeywordTok{as.data.frame}\NormalTok{(PTdtm)}

\CommentTok{#add sentiment target to the data frame}
\NormalTok{PTdtmDF_sa <-}\StringTok{ }\KeywordTok{cbind}\NormalTok{(PTdtmDF, }\DataTypeTok{sentiment_target =} \NormalTok{TwitTerData_PT_Processed$sentiment)}

\CommentTok{#confirm there are not Na values in the dataset}
\KeywordTok{anyNA}\NormalTok{(PTdtmDF_sa)}

\CommentTok{#save EN_dtm as a RDS}
\KeywordTok{saveRDS} \NormalTok{(PTdtmDF_sa, }\StringTok{"../data/labeled_pt/PTdtmDF_sa.RDS"}\NormalTok{)}
\end{Highlighting}
\end{Shaded}

After converting the dataset to a document text matrix, the next step is
to start building the classification model. The classification model
used in this project is based on the naive Bayes logistic algorithm.
Because of machine configuration, we will only use 70\% of the data and
split it in 80/20 for training and testing.

\begin{Shaded}
\begin{Highlighting}[]
\NormalTok{PTdtmDF_sa <-}\StringTok{ }\KeywordTok{readRDS}\NormalTok{(}\StringTok{"../data/labeled_pt/PTdtmDF_sa.RDS"}\NormalTok{) ## - > start here if you only want to run the model}


\CommentTok{#Filter out the dataset with the tweets extracted linked to the venezuelan conflict }
\NormalTok{holdout_pt <-}\StringTok{ }\KeywordTok{is.na}\NormalTok{(PTdtmDF_sa$sentiment_target)}

\NormalTok{PTdtmDF_sa <-}\StringTok{ }\NormalTok{PTdtmDF_sa[!holdout_pt, ]}


\CommentTok{# Extract sample of the dataset   }
\KeywordTok{set.seed}\NormalTok{(}\DecValTok{9876}\NormalTok{)}
\NormalTok{PT_index <-}\StringTok{ }\NormalTok{caret::}\KeywordTok{createDataPartition}\NormalTok{(PTdtmDF_sa$sentiment_target, }\DataTypeTok{p =} \FloatTok{0.7}\NormalTok{, }
                                       \DataTypeTok{list =} \OtherTok{FALSE}\NormalTok{, }
                                       \DataTypeTok{times =} \DecValTok{1}\NormalTok{)}

\NormalTok{PTdtmDF_sample <-}\StringTok{ }\NormalTok{PTdtmDF_sa[PT_index, ]}

\CommentTok{#confirm that the number of classification variables is distributed equally}
\KeywordTok{table}\NormalTok{(PTdtmDF_sample$sentiment)}
\end{Highlighting}
\end{Shaded}

Now we have the sample consisting of 172,524 negative tweets and 173,320
positive tweets. This dataset will be split in training and testing.

\begin{Shaded}
\begin{Highlighting}[]
\CommentTok{#Split dataset in training (80%) and testing (20%)}
\KeywordTok{set.seed}\NormalTok{(}\DecValTok{2312}\NormalTok{)}
\NormalTok{PT_index_sample <-}\StringTok{ }\NormalTok{caret::}\KeywordTok{createDataPartition}\NormalTok{(PTdtmDF_sample$sentiment, }\DataTypeTok{p =} \FloatTok{0.8}\NormalTok{, }
                                             \DataTypeTok{list =} \OtherTok{FALSE}\NormalTok{, }
                                             \DataTypeTok{times =} \DecValTok{1}\NormalTok{)      }

\NormalTok{PT_training <-}\StringTok{ }\NormalTok{PTdtmDF_sample[PT_index_sample, ]}
\NormalTok{PT_test  <-}\StringTok{ }\NormalTok{PTdtmDF_sample[-PT_index_sample,]}

\CommentTok{#verify the distribution of the dataset }
\KeywordTok{table}\NormalTok{(PT_training$sentiment_target)}

    \CommentTok{#save EN_dtm as a RDS}
    \KeywordTok{saveRDS} \NormalTok{(PT_training, }\StringTok{"../data/labeled_pt/PT_training.RDS"}\NormalTok{)}
    \KeywordTok{saveRDS} \NormalTok{(PT_test, }\StringTok{"../data/labeled_pt/PT_test.RDS"}\NormalTok{)}
\end{Highlighting}
\end{Shaded}

The final step is to run the classification model, using the Bayesian
Generalized Linear Models for supervised machine learning.

\begin{Shaded}
\begin{Highlighting}[]
\CommentTok{#Need to increase memory to }
\KeywordTok{memory.limit}\NormalTok{(}\DataTypeTok{size=} \DecValTok{53000}\NormalTok{)    }

\NormalTok{PT_training <-}\StringTok{ }\KeywordTok{readRDS}\NormalTok{(}\StringTok{"../data/labeled_pt/PT_training.RDS"}\NormalTok{)}


\CommentTok{# Training the model}
\NormalTok{StartTIme <-}\StringTok{ }\KeywordTok{Sys.time}\NormalTok{()}
\NormalTok{fit_pt <-}\StringTok{ }\NormalTok{caret::}\KeywordTok{train}\NormalTok{(sentiment_target ~}\StringTok{ }\NormalTok{., }\DataTypeTok{data =} \NormalTok{PT_training, }\DataTypeTok{method =} \StringTok{'bayesglm'}\NormalTok{)}
\NormalTok{EndTime <-}\StringTok{ }\KeywordTok{Sys.time}\NormalTok{()}
\NormalTok{EndTime -}\StringTok{ }\NormalTok{StartTIme}


\KeywordTok{saveRDS} \NormalTok{(fit_pt, }\StringTok{"../data/labeled_pt/fit_pt.RDS"}\NormalTok{)}
\end{Highlighting}
\end{Shaded}

And test the model.

\begin{Shaded}
\begin{Highlighting}[]
\CommentTok{#Need to increase memory to run the process in my machine}
\KeywordTok{memory.limit}\NormalTok{(}\DataTypeTok{size=} \DecValTok{53000}\NormalTok{)    }

\NormalTok{PT_test <-}\StringTok{ }\KeywordTok{readRDS} \NormalTok{(}\StringTok{"../data/labeled_pt/PT_test.RDS"}\NormalTok{)}
\NormalTok{fit_pt <-}\StringTok{ }\KeywordTok{readRDS}\NormalTok{(}\StringTok{"../data/labeled_pt/fit_pt.RDS"}\NormalTok{)}



\CommentTok{# Testing the model}
\NormalTok{predict_pt <-}\StringTok{ }\KeywordTok{predict}\NormalTok{(fit_pt, }\DataTypeTok{newdata =} \NormalTok{PT_test)}
\end{Highlighting}
\end{Shaded}

\begin{Shaded}
\begin{Highlighting}[]
\CommentTok{# Verify the results}
\NormalTok{fit_pt }

\CommentTok{#confirm the results}
\NormalTok{expected_PT <-}\StringTok{ }\NormalTok{PT_test$sentiment_target}

\NormalTok{results_PT <-}\StringTok{ }\KeywordTok{confusionMatrix}\NormalTok{(}\DataTypeTok{data =} \NormalTok{predict_pt, }\DataTypeTok{reference =} \NormalTok{expected_PT)}

\NormalTok{results_PT}
\end{Highlighting}
\end{Shaded}

**** Insert here classification results *******

\subsubsection{4. Predictions and analysis of
results}\label{predictions-and-analysis-of-results}

With the classification models created in the previous phase of the
project, we now apply the model to the data extraction for each
language.

\paragraph{4.1 Prediction for Tweets in
English}\label{prediction-for-tweets-in-english}

\begin{Shaded}
\begin{Highlighting}[]
\NormalTok{ENdtmDF_sa <-}\StringTok{ }\KeywordTok{readRDS} \NormalTok{(}\StringTok{"../data/labeled_en/ENdtmDF_sa.RDS"}\NormalTok{)}

\CommentTok{#get the extracted data to run the classiffier}
\NormalTok{en_vzla_holdout <-}\StringTok{ }\NormalTok{ENdtmDF_sa[holdout_en, ]}

\CommentTok{#Run the classification model}
\NormalTok{en_vzla_prediction <-}\StringTok{ }\KeywordTok{predict}\NormalTok{(fit, }\DataTypeTok{newdata =} \NormalTok{en_vzla_holdout)}


\CommentTok{#Add the classification to the tweets}

\NormalTok{en_vzla_predictionDF <-}\StringTok{ }\KeywordTok{cbind}\NormalTok{(en_vzla_holdout, en_vzla_prediction)}

\NormalTok{en_vzla_predictionDF <-}\StringTok{ }\NormalTok{en_vzla_predictionDF%>%}\StringTok{ }
\StringTok{  }\NormalTok{tibble::}\KeywordTok{rownames_to_column}\NormalTok{(}\DataTypeTok{var =} \StringTok{"rowid"}\NormalTok{) %>%}\StringTok{ }
\StringTok{  }\NormalTok{dplyr::}\KeywordTok{select}\NormalTok{(rowid, en_vzla_prediction)}


\NormalTok{TwitTerData_EN_Processed_vla <-}\StringTok{ }\NormalTok{TwitTerData_EN_Processed[holdout_en, ] %>%}\StringTok{ }
\StringTok{  }\NormalTok{tibble::}\KeywordTok{rownames_to_column}\NormalTok{(}\DataTypeTok{var =} \StringTok{"rowid"}\NormalTok{)}

\NormalTok{en_vzla_predictionDF_tweets <-}\StringTok{ }\NormalTok{en_vzla_predictionDF %>%}\StringTok{ }
\StringTok{  }\KeywordTok{left_join}\NormalTok{(TwitTerData_EN_Processed_vla, }\DataTypeTok{by =} \StringTok{"rowid"}\NormalTok{) %>%}\StringTok{ }
\StringTok{  }\KeywordTok{select}\NormalTok{(id, text, text_sa, }\DataTypeTok{sentiment =} \NormalTok{en_vzla_prediction)}
\end{Highlighting}
\end{Shaded}

\begin{Shaded}
\begin{Highlighting}[]
\CommentTok{#Get results of the predictions and compare some results}
\KeywordTok{table}\NormalTok{(en_vzla_predictionDF_tweets$sentiment)}


\CommentTok{#See some results of the negative classification}
\NormalTok{negative_en <-}\StringTok{ }\NormalTok{en_vzla_predictionDF_tweets %>%}\StringTok{ }
\StringTok{  }\NormalTok{dplyr::}\KeywordTok{filter}\NormalTok{(sentiment ==}\StringTok{ }\DecValTok{0}\NormalTok{) %>%}\StringTok{ }
\StringTok{  }\KeywordTok{select}\NormalTok{(text, text_sa)}

\KeywordTok{tail}\NormalTok{(negative_en, }\DataTypeTok{n=}\DecValTok{20}\NormalTok{)}

\CommentTok{#Test results with some of the words}
\NormalTok{negative_en_words <-}\StringTok{ }\NormalTok{en_vzla_predictionDF_tweets %>%}\StringTok{ }
\StringTok{  }\NormalTok{dplyr::}\KeywordTok{filter}\NormalTok{(}\KeywordTok{grepl}\NormalTok{(}\StringTok{'bad|harm'}\NormalTok{, text)) %>%}\StringTok{ }
\StringTok{  }\KeywordTok{select}\NormalTok{(text, text_sa, sentiment)}

\CommentTok{#see how tweets with these words were classified}
\KeywordTok{table}\NormalTok{(negative_en_words$sentiment)}

\NormalTok{negative_en_words_test <-}\StringTok{ }\NormalTok{negative_en_words %>%}\StringTok{ }
\StringTok{  }\NormalTok{dplyr::}\KeywordTok{filter}\NormalTok{(sentiment ==}\StringTok{ }\DecValTok{1}\NormalTok{)}
\end{Highlighting}
\end{Shaded}

\paragraph{4.2 Prediction for Tweets in
Portuguese}\label{prediction-for-tweets-in-portuguese}

\begin{Shaded}
\begin{Highlighting}[]
\NormalTok{PTdtmDF_sa <-}\StringTok{ }\KeywordTok{readRDS}\NormalTok{(}\StringTok{"../data/labeled_pt/PTdtmDF_sa.RDS"}\NormalTok{)}


\CommentTok{#Filter out the dataset with the tweets extracted linked to the venezuelan conflict }
\NormalTok{holdout_pt <-}\StringTok{ }\KeywordTok{is.na}\NormalTok{(PTdtmDF_sa$sentiment_target)}


\CommentTok{#get the extracted data to run the classiffier}
\NormalTok{pt_vzla_holdout <-}\StringTok{ }\NormalTok{PTdtmDF_sa[holdout_pt, ]}

\CommentTok{#Run the classification model}
\NormalTok{pt_vzla_prediction <-}\StringTok{ }\KeywordTok{predict}\NormalTok{(fit_pt, }\DataTypeTok{newdata =} \NormalTok{pt_vzla_holdout)}


\CommentTok{#Add the classification to the tweets}

\NormalTok{pt_vzla_predictionDF <-}\StringTok{ }\KeywordTok{cbind}\NormalTok{(pt_vzla_holdout, pt_vzla_prediction)}

\NormalTok{pt_vzla_predictionDF <-}\StringTok{ }\NormalTok{pt_vzla_predictionDF%>%}\StringTok{ }
\StringTok{  }\NormalTok{tibble::}\KeywordTok{rownames_to_column}\NormalTok{(}\DataTypeTok{var =} \StringTok{"rowid"}\NormalTok{) %>%}\StringTok{ }
\StringTok{  }\NormalTok{dplyr::}\KeywordTok{select}\NormalTok{(rowid, pt_vzla_prediction)}


\NormalTok{TwitTerData_PT_Processed_vla <-}\StringTok{ }\NormalTok{TwitTerData_PT_Processed[holdout_pt, ] %>%}\StringTok{ }
\StringTok{  }\NormalTok{tibble::}\KeywordTok{rownames_to_column}\NormalTok{(}\DataTypeTok{var =} \StringTok{"rowid"}\NormalTok{)}

\KeywordTok{str}\NormalTok{(TwitTerData_PT_Processed_vla)}

\NormalTok{pt_vzla_predictionDF_tweets <-}\StringTok{ }\NormalTok{pt_vzla_predictionDF %>%}\StringTok{ }
\StringTok{  }\KeywordTok{left_join}\NormalTok{(TwitTerData_PT_Processed_vla, }\DataTypeTok{by =} \StringTok{"rowid"}\NormalTok{) %>%}\StringTok{ }
\StringTok{  }\KeywordTok{select}\NormalTok{(id, tweet_text_, text_sa, }\DataTypeTok{sentiment =} \NormalTok{pt_vzla_prediction)}
\end{Highlighting}
\end{Shaded}

\begin{Shaded}
\begin{Highlighting}[]
\CommentTok{#Get results of the predictions and compare some results}
\KeywordTok{table}\NormalTok{(pt_vzla_predictionDF_tweets$sentiment)}


\CommentTok{#See some results of the negative classification}
\NormalTok{negative_pt <-}\StringTok{ }\NormalTok{pt_vzla_predictionDF_tweets %>%}\StringTok{ }
\StringTok{  }\NormalTok{dplyr::}\KeywordTok{filter}\NormalTok{(sentiment ==}\StringTok{ }\DecValTok{0}\NormalTok{) %>%}\StringTok{ }
\StringTok{  }\KeywordTok{select}\NormalTok{(tweet_text_, text_sa) %>%}\StringTok{ }\NormalTok{collect}

\KeywordTok{head}\NormalTok{(negative_pt, }\DataTypeTok{n=}\DecValTok{20}\NormalTok{)}

\CommentTok{#Test results with some of the words}
\NormalTok{negative_pt_words <-}\StringTok{ }\NormalTok{pt_vzla_predictionDF_tweets %>%}\StringTok{ }
\StringTok{  }\NormalTok{dplyr::}\KeywordTok{filter}\NormalTok{(}\KeywordTok{grepl}\NormalTok{(}\StringTok{'miserável|ditador'}\NormalTok{, tweet_text_)) %>%}\StringTok{ }
\StringTok{  }\KeywordTok{select}\NormalTok{(tweet_text_, text_sa, sentiment) %>%}\StringTok{ }\NormalTok{collect}

\CommentTok{#see how tweets with these words were classified}
\KeywordTok{table}\NormalTok{(negative_pt_words$sentiment)}

\NormalTok{negative_en_words_test <-}\StringTok{ }\NormalTok{negative_en_words %>%}\StringTok{ }
\StringTok{  }\NormalTok{dplyr::}\KeywordTok{filter}\NormalTok{(sentiment ==}\StringTok{ }\DecValTok{1}\NormalTok{)}
\end{Highlighting}
\end{Shaded}

\subsubsection{5. Lessons learned and next
steps}\label{lessons-learned-and-next-steps}

\begin{Shaded}
\begin{Highlighting}[]
\KeywordTok{sessionInfo}\NormalTok{()}
\end{Highlighting}
\end{Shaded}

\begin{verbatim}
## R version 3.5.3 (2019-03-11)
## Platform: x86_64-w64-mingw32/x64 (64-bit)
## Running under: Windows 10 x64 (build 17763)
## 
## Matrix products: default
## 
## locale:
## [1] LC_COLLATE=English_Canada.1252  LC_CTYPE=English_Canada.1252   
## [3] LC_MONETARY=English_Canada.1252 LC_NUMERIC=C                   
## [5] LC_TIME=English_Canada.1252    
## 
## attached base packages:
## [1] stats     graphics  grDevices utils     datasets  methods   base     
## 
## other attached packages:
##  [1] plyr_1.8.4      purrr_0.3.2     twitteR_1.1.9   tm_0.7-6       
##  [5] NLP_0.2-0       magrittr_1.5    dplyr_0.8.0.1   caret_6.0-82   
##  [9] ggplot2_3.1.1   lattice_0.20-38
## 
## loaded via a namespace (and not attached):
##  [1] tidyselect_0.2.5   xfun_0.6           slam_0.1-45       
##  [4] reshape2_1.4.3     splines_3.5.3      colorspace_1.4-1  
##  [7] generics_0.0.2     htmltools_0.3.6    stats4_3.5.3      
## [10] yaml_2.2.0         survival_2.43-3    prodlim_2018.04.18
## [13] rlang_0.3.4        ModelMetrics_1.2.2 pillar_1.3.1      
## [16] DBI_1.0.0          glue_1.3.1         withr_2.1.2       
## [19] bit64_0.9-7        foreach_1.4.4      lava_1.6.5        
## [22] stringr_1.4.0      timeDate_3043.102  munsell_0.5.0     
## [25] gtable_0.3.0       recipes_0.1.5      codetools_0.2-16  
## [28] evaluate_0.13      knitr_1.22         parallel_3.5.3    
## [31] class_7.3-15       Rcpp_1.0.1         scales_1.0.0      
## [34] ipred_0.9-8        bit_1.1-14         rjson_0.2.20      
## [37] digest_0.6.18      stringi_1.4.3      grid_3.5.3        
## [40] tools_3.5.3        lazyeval_0.2.2     tibble_2.1.1      
## [43] crayon_1.3.4       pkgconfig_2.0.2    MASS_7.3-51.1     
## [46] Matrix_1.2-15      xml2_1.2.0         data.table_1.12.2 
## [49] lubridate_1.7.4    gower_0.2.0        httr_1.4.0        
## [52] assertthat_0.2.1   rmarkdown_1.12     iterators_1.0.10  
## [55] R6_2.4.0           rpart_4.1-13       nnet_7.3-12       
## [58] nlme_3.1-137       compiler_3.5.3
\end{verbatim}

\subsubsection{References}\label{references}

{[}1{]} Go, A., Bhayani, R. and Huang, L., 2009. Twitter sentiment
classification using distant supervision. CS224N Project Report,
Stanford, 1(2009), p.12. Retrieved from
\url{https://www.kaggle.com/kazanova/sentiment140}"

{[}2{]} de Santiago, R. Portuguese tweets for sentiment analysis.
Retrieved from
\url{https://www.kaggle.com/augustop/portuguese-tweets-for-sentiment-analysis}


\end{document}
